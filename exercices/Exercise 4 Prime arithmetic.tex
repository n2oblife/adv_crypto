\documentclass{article}
\usepackage[margin=1.5cm]{geometry}
\usepackage[parfill]{parskip}
\usepackage[utf8]{inputenc}
\usepackage{amsmath,amssymb,amsfonts,amsthm}
\author{Add your name here}
\date{XX.XX.2025}
\title{Prime arithmetic}

\usepackage{filecontents}

\begin{filecontents}{lecture-tum-aci-ss2025-U4.bib}
@book{HAC,
  added-at = {2007-04-16T14:16:26.000+0200},
  author = {Menezes, Alfred J. and van Oorschot, Paul C. and Vanstone, Scott A.},
  biburl = {https://www.bibsonomy.org/bibtex/21fb86a3afb62bbb4bc2f94a0299a4d59/kidloc},
  interhash = {d5837a554ce2ffd49f551045400e244c},
  intrahash = {1fb86a3afb62bbb4bc2f94a0299a4d59},
  keywords = {cryptography handbook programming reference security},
  publisher = {CRC Press},
  timestamp = {2007-04-16T14:16:26.000+0200},
  title = {Handbook of Applied Cryptography},
  url = {http://www.cacr.math.uwaterloo.ca/hac/},
  year = 2001
}
\end{filecontents}

\usepackage{biblatex}
\addbibresource{lecture-tum-aci-ss2025-U4.bib}
\nocite{*}

\begin{document}
\maketitle

\section{Exercise \#1}

1. Write the operand scanning multiplication algorithm (Alg. 12) for $m=n=3$.

2. Provide a graphical representation of the operand scanning multiplication algorithm  (Alg. 12) for $m=n=3$: represent the sequence how the partial products
$a_ib_j$ are computed and added together including the carries (Line 6, Alg. 12).

\section{Exercise \#2}

1. Write the operand scanning multiplication algorithm (Alg. 12) for $m=n=3$ and $A=B$.

2. Optimize the operand scanning multiplication algorithm (Alg. 12) when $A=B$: write the optimized algorithm.

3. Provide a graphical representation of the optimized operand scanning multiplication algorithm  (Alg. 12) when $A=B$ and $m=n=3$.

\section{Exercise \#3 (Extra)}

1. Explain the proof of termination of Alg. 14 (cf. \cite[p. 605]{HAC}).

2. Explain the proof of correctness of Alg. 14 (cf. \cite[p. 605]{HAC}).

\printbibliography


\end{document}
