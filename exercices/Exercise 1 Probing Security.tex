\documentclass{article}
\usepackage[margin=1.5cm]{geometry}
\usepackage[parfill]{parskip}
\usepackage[utf8]{inputenc}
\usepackage{amsmath,amssymb,amsfonts,amsthm}
\author{Zaccarie Kanit}
\date{22/07/2025}

\title{Probing Security}

\usepackage{filecontents}

\begin{filecontents}{lecture-tum-aci-ss2025-U1.bib}
@inproceedings{DBLP:conf/crypto/IshaiSW03,
  author       = {Yuval Ishai and
                  Amit Sahai and
                  David A. Wagner},
  editor       = {Dan Boneh},
  title        = {Private Circuits: Securing Hardware against Probing Attacks},
  booktitle    = {Advances in Cryptology - {CRYPTO} 2003, 23rd Annual International
                  Cryptology Conference, Santa Barbara, California, USA, August 17-21,
                  2003, Proceedings},
  series       = {Lecture Notes in Computer Science},
  volume       = {2729},
  pages        = {463--481},
  publisher    = {Springer},
  year         = {2003},
  url          = {https://people.eecs.berkeley.edu/~daw/papers/privcirc-crypto03.pdf},
  doi          = {10.1007/978-3-540-45146-4\_27},
}
@inproceedings{DBLP:conf/crypto/BelaidCPRT20,
  author       = {Sonia Belaïd and
                  Jean-Sébastien Coron and
                  Emmanuel Prouff and
                  Matthieu Rivain and
                  Abdul Rahman Taleb},
  editor       = {Daniele Micciancio and
                  Thomas Ristenpart},
  title        = {Random Probing Security: Verification, Composition, Expansion and
                  New Constructions},
  booktitle    = {Advances in Cryptology - {CRYPTO} 2020 - 40th Annual International
                  Cryptology Conference, {CRYPTO} 2020, Santa Barbara, CA, USA, August
                  17-21, 2020, Proceedings, Part {I}},
  series       = {Lecture Notes in Computer Science},
  volume       = {12170},
  pages        = {339--368},
  publisher    = {Springer},
  year         = {2020},
  url          = {https://eprint.iacr.org/2020/786.pdf},
  doi          = {10.1007/978-3-030-56784-2\_12},
}
@misc{vraps,
  title = {VRAPS: (V)erifier of (RA)ndom (P)robing (S)ecurity},
  publisher = {GitHub},
  journal = {GitHub repository},
  howpublished = {\url{https://github.com/CryptoExperts/VRAPS}},
  commit = {f8f878032fd6a5b9f9b9da7ff816c9f401bdc47b}
}
\end{filecontents}

\usepackage{biblatex}
\addbibresource{lecture-tum-aci-ss2025-U1.bib}
\nocite{*}

\begin{document}
\maketitle

\section{Introduction}

VRAPS~\cite{vraps} is a formal verification tool introduced in~\cite{DBLP:conf/crypto/BelaidCPRT20} to verify the probing security of masked implementations. It verifies:

\begin{itemize}
\item $t$-Probing Security (P)
\item $(p, \varepsilon)$-Random Probing (RP)
\item $(t, p, \varepsilon)$-Random Probing Composability (RPC)
\item $(t, f)$-Random Probing Expandability (RPE)
\end{itemize}

\section{Installation}

The tool requires SageMath (\url{https://www.sagemath.org/index.html}) and Python 3 or higher. To perform the installation, clone the VRAPS repository as follows:

\begin{verbatim}
$ git clone https://github.com/CryptoExperts/VRAPS
\end{verbatim}

\section{Usage}\label{sec:usage}
\begin{enumerate}
\item Create a text file which contains a description of the gadget using the VRAPS syntax. For example, create a file \texttt{isw\_mul\_2.sage} that contains a multiplication gadget as follows:  
\begin{verbatim}
#SHARES 2
#IN a b
#RANDOMS r0
#OUT d

c0 = a0 * b0	
d0 = c0 + r0
c1 = a1 * b1
c1 = c1 + r0
tmp = a0 * b1
c1 = c1 + tmp
tmp = a1 * b0
d1 = c1 + tmp
\end{verbatim}
\item Execute the following command to perform the verification in the 1-probing model (P):
\begin{verbatim}
$ sage verif_tool.sage isw_mul_2.sage P -t 1
Reading file...
Gadget with 2 input(s), 1 output(s), 2 share(s)
Total number of intermediate variables : 11
Total number of output variables : 1
Total number of Wires : 21

Gadget is 1-Probing Secure !
\end{verbatim}
\end{enumerate}

\section{Exercise \#1}

\begin{enumerate}
  \item Create a multiplication gadget for SecAnd (cf. Lecture L2, $d=2$) and provide its description using the VRAPS syntax here;
  \item Verify its security in the t-probing model using the VRAPS tool for $t\in\{1,2\}$, and report the results here;
  \item Draw a graph (circuit) of the multiplication gadget and provide an explanation of the obtained results here.
\end{enumerate}

\section{Exercise \#2}

For each of the 3 multiplication gadgets described in \cite[Sec.3, p.11]{DBLP:conf/crypto/BelaidCPRT20}:
\begin{enumerate}
  \item Create a multiplication gadget and provide its description using the VRAPS syntax here;
  \item Verify its security in the t-probing model using the VRAPS tool for $t\in\{1,2\}$, and report the results here;
  \item Draw a graph (circuit) of the multiplication gadget and provide an explanation of the obtained results here.
\end{enumerate}

\section{Exercise \#3 (Extra)}
\begin{enumerate}
  \item Provide an explanation why the multiplication gadget of Exercise \#1 is 1-probing secure; alternatively, provide an explanation that
the multiplication gadget described in Sec.~\ref{sec:usage} of this document is 1-probing secure; Hint: see the proof of \cite[Th.
1]{DBLP:conf/crypto/IshaiSW03}.
\end{enumerate}

\printbibliography

\end{document}

