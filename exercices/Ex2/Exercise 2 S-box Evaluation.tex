\documentclass{article}
\usepackage[margin=1.5cm]{geometry}
\usepackage[parfill]{parskip}
\usepackage[utf8]{inputenc}
\usepackage{amsmath,amssymb,amsfonts,amsthm}

\newtheorem{definition}{Definition}

\author{Zaccarie Kanit}
\date{22/07/2025}

\title{S-box evaluation methods}

\usepackage{filecontents}

\begin{filecontents}{lecture-tum-aci-ss2024-U2.bib}
\end{filecontents}

\usepackage{biblatex}
\addbibresource{lecture-tum-aci-ss2024-U2.bib}
\nocite{*}

\begin{document}
\maketitle

\section*{Exercise \#1}
Find three distinct addition chains to evaluate $x^{254}$, describe the employed methodology, 
and discuss the results.

\subsection*{Response}
\begin{enumerate}
  \item Binary Method (Square-and-Multiply), convert 254 to binary and use repeated squaring
  with selective multiplications at the end (square number of time there is a 1 until first 0 i binary),
  $254_{10} = 11111110_2$. \\ 
  Reliable and systematic but not optimal. Good baseline method : \\
  1 →[+1] 2 →[+2] 4 →[+4] 8 →[+8] 16 →[+16] 32 →[+32] 64 →[+64] 128 →[+64] 192 →[+32] 224 →[+16] 240 →[+8] 248 →[+4] 252 →[+2] 254
  % 1
  % 2 = 1 + 1
  % 4 = 2 + 2
  % 8 = 4 + 4
  % 16 = 8 + 8
  % 32 = 16 + 16
  % 64 = 32 + 32
  % 128 = 64 + 64
  % 192 = 128 + 64
  % 224 = 192 + 32
  % 240 = 224 + 16
  % 248 = 240 + 8
  % 252 = 248 + 4
  % 254 = 252 + 2
  \item Sliding Window Method, use precomputed small powers and sliding window technique. \\
  The sliding window method produces the shortest chain (11 operations) by cleverly reusing intermediate results. : \\
  1 →[+1] 2 →[+1] 3 →[+1] 4 →[+3] 7 →[+7] 14 →[+14] 28 →[+28] 56 →[+7] 63 →[+63] 126 →[+126] 252 →[+2] 254
  % 1
  % 2 = 1 + 1
  % 3 = 2 + 1
  % 4 = 2 + 2
  % 7 = 4 + 3
  % 14 = 7 + 7
  % 28 = 14 + 14
  % 56 = 28 + 28
  % 63 = 56 + 7
  % 126 = 63 + 63
  % 252 = 126 + 126
  % 254 = 252 + 2
  \item Factor-Based Method, decompose 254 using its factorization and build chains for factors
  ( $254 = 2 \times 127$ ). \\
  Less effective here because 127 is prime, but would be excellent for highly composite exponents. : \\
  1 →[+1] 2 →[+2] 4 →[+4] 8 →[+8] 16 →[+16] 32 →[+32] 64 →[+32] 96 →[+16] 112 →[+8] 120 →[+4] 124 →[+2] 126 →[+1] 127 →[+127] 254
  % 1
  % 2 = 1 + 1
  % 4 = 2 + 2
  % 8 = 4 + 4
  % 16 = 8 + 8
  % 32 = 16 + 16
  % 64 = 32 + 32
  % 96 = 64 + 32
  % 112 = 96 + 16
  % 120 = 112 + 8
  % 124 = 120 + 4
  % 126 = 124 + 2
  % 127 = 126 + 1
  % 254 = 127 + 127
\end{enumerate}

\section*{Exercise \#2}

Let $C_\alpha = \{ \alpha2^i \bmod{2^n-1}, i\in[0,n-1]\}$ denote the cyclotomic class of $\alpha\in[0,2^n-2]$.

Let $\mathcal{M}_k^n$ denote the class of exponents $\alpha$ such that $x^\alpha$ has a masking complexity equal to $k$.

Compute the cyclotomic and exponent classes for $n=5$.

\subsection*{Response}

For $n=5$, we work in the field $\mathbb{F}_{2^5}$ with modulus $2^5-1 = 31$.
\begin{definition}[Cyclotomic Class]
$C_\alpha = \{ \alpha \cdot 2^i \bmod{31} : i \in [0,4]\}$
\end{definition}

\begin{definition}[Masking Complexity Classes]
$\mathcal{M}_k^5 = \{\alpha : \text{masking complexity of } x^\alpha \text{ equals } k\}$
\end{definition}

\subsubsection*{Computing All Cyclotomic Classes}

The cyclotomic classes partition $\{0, 1, 2, \ldots, 30\}$ into equivalence classes under multiplication by powers of 2 modulo 31.

\subsubsection*{Class Computations}

\textbf{Trivial Class:} $C_0 = \{0\}$

\textbf{Class $C_1$:}
\begin{align*}
1 \cdot 2^0 &= 1 \pmod{31} \\
1 \cdot 2^1 &= 2 \pmod{31} \\
1 \cdot 2^2 &= 4 \pmod{31} \\
1 \cdot 2^3 &= 8 \pmod{31} \\
1 \cdot 2^4 &= 16 \pmod{31}
\end{align*}
Therefore, $C_1 = \{1, 2, 4, 8, 16\}$.

\textbf{Class $C_3$:}
\begin{align*}
3 \cdot 2^0 &= 3 \pmod{31} \\
3 \cdot 2^1 &= 6 \pmod{31} \\
3 \cdot 2^2 &= 12 \pmod{31} \\
3 \cdot 2^3 &= 24 \pmod{31} \\
3 \cdot 2^4 &= 48 = 17 \pmod{31}
\end{align*}
Therefore, $C_3 = \{3, 6, 12, 17, 24\}$.

\textbf{Class $C_5$:}
\begin{align*}
5 \cdot 2^0 &= 5 \pmod{31} \\
5 \cdot 2^1 &= 10 \pmod{31} \\
5 \cdot 2^2 &= 20 \pmod{31} \\
5 \cdot 2^3 &= 40 = 9 \pmod{31} \\
5 \cdot 2^4 &= 80 = 18 \pmod{31}
\end{align*}
Therefore, $C_5 = \{5, 9, 10, 18, 20\}$.

\textbf{Class $C_7$:}
\begin{align*}
7 \cdot 2^0 &= 7 \pmod{31} \\
7 \cdot 2^1 &= 14 \pmod{31} \\
7 \cdot 2^2 &= 28 \pmod{31} \\
7 \cdot 2^3 &= 56 = 25 \pmod{31} \\
7 \cdot 2^4 &= 112 = 19 \pmod{31}
\end{align*}
Therefore, $C_7 = \{7, 14, 19, 25, 28\}$.

\textbf{Class $C_{11}$:}
\begin{align*}
11 \cdot 2^0 &= 11 \pmod{31} \\
11 \cdot 2^1 &= 22 \pmod{31} \\
11 \cdot 2^2 &= 44 = 13 \pmod{31} \\
11 \cdot 2^3 &= 88 = 26 \pmod{31} \\
11 \cdot 2^4 &= 176 = 21 \pmod{31}
\end{align*}
Therefore, $C_{11} = \{11, 13, 21, 22, 26\}$.

\textbf{Class $C_{15}$:}
\begin{align*}
15 \cdot 2^0 &= 15 \pmod{31} \\
15 \cdot 2^1 &= 30 \pmod{31} \\
15 \cdot 2^2 &= 60 = 29 \pmod{31} \\
15 \cdot 2^3 &= 120 = 27 \pmod{31} \\
15 \cdot 2^4 &= 240 = 23 \pmod{31}
\end{align*}
Therefore, $C_{15} = \{15, 23, 27, 29, 30\}$.

\subsubsection*{Complete Partition}

The complete set of cyclotomic classes for $n=5$ is:
\begin{align*}
C_0 &= \{0\} \\
C_1 &= \{1, 2, 4, 8, 16\} \\
C_3 &= \{3, 6, 12, 17, 24\} \\
C_5 &= \{5, 9, 10, 18, 20\} \\
C_7 &= \{7, 14, 19, 25, 28\} \\
C_{11} &= \{11, 13, 21, 22, 26\} \\
C_{15} &= \{15, 23, 27, 29, 30\}
\end{align*}

\textbf{Verification:} $1 + 5 + 5 + 5 + 5 + 5 + 5 = 31 = 2^5 - 1$ \checkmark

\subsubsection*{Masking Complexity Analysis}

The masking complexity of $x^\alpha$ is determined by the minimum number of multiplications required to compute $x^\alpha$ using a secure masking scheme resistant to side-channel attacks.

\subsubsection*{Complexity Classification}

\textbf{Zero Complexity:}
$\mathcal{M}_0^5 = \{0\}$
Since $x^0 = 1$ requires no multiplications.

\textbf{Single Multiplication:}
$\mathcal{M}_1^5 = \{1, 2, 4, 8, 16\} = C_1$
Powers of 2 can be computed with a single squaring operation. Each element requires exactly 1 multiplication.

\textbf{Two Multiplications:}
For $\alpha = 3$: $x^3 = x \cdot x^2$ (2 operations), so $\mathcal{M}_2^5 \supseteq C_3$.

For $\alpha = 5$: $x^5 = x \cdot x^4$ (2 operations), so $\mathcal{M}_2^5 \supseteq C_5$.

\textbf{Three Multiplications:}
For $\alpha = 7$: $x^7 = x^3 \cdot x^4$ or $x^7 = x \cdot x^2 \cdot x^4$ (3 operations), so $\mathcal{M}_3^5 \supseteq C_7$.

For $\alpha = 11$: $x^{11} = x^3 \cdot x^8$ (3 operations), so $\mathcal{M}_3^5 \supseteq C_{11}$.

For $\alpha = 15$: $x^{15} = x^3 \cdot x^4 \cdot x^8$ or optimal chains (3 operations), so $\mathcal{M}_3^5 \supseteq C_{15}$.

\subsubsection*{Final Classification}

The masking complexity classes for $n=5$ are:
\begin{align*}
\mathcal{M}_0^5 &= \{0\} \\
\mathcal{M}_1^5 &= \{1, 2, 4, 8, 16\} \\
\mathcal{M}_2^5 &= \{3, 5, 6, 9, 10, 12, 17, 18, 20, 24\} \\
\mathcal{M}_3^5 &= \{7, 11, 13, 14, 15, 19, 21, 22, 23, 25, 26, 27, 28, 29, 30\}
\end{align*}

\subsubsection*{Key Observations}

\begin{enumerate}
\item \textbf{Class Preservation:} Elements in the same cyclotomic class have identical masking complexity.
\item \textbf{Optimal Chains:} The complexity is determined by optimal addition chains for exponentiation.
\item \textbf{Security Implications:} Higher complexity classes require more sophisticated masking techniques.
\item \textbf{Field Structure:} The cyclotomic class structure reflects the multiplicative structure of $\mathbb{F}_{2^5}^*$.
\end{enumerate}

\subsubsection*{Computational Verification}

All cyclotomic classes are closed under the operation $\alpha \mapsto 2\alpha \bmod{31}$, and the masking complexities are consistent with known results for secure exponentiation algorithms in characteristic 2 finite fields.

\section*{Exercise \#3 (Extra)}

Assume the following operations are available (in increasing order of computational effort):
\begin{itemize}
  \item Type-I: $f(x)=x^{2^\ell}$;
  \item Type-II: $g(x)=xx^{2^\ell}$;
  \item Type-III: $h(x,y)=xy$.
\end{itemize}


1. Find a sequence of operations that evaluate $x^{254}$ with maximum three Type-II operations.

2. Find a sequence of operations that evaluate $x^{254}$ with less than four Type-III operations.

3. Describe an algorithm or a methodology to find a sequence of operations that evaluates a given monomial $x^{\alpha}$ with the minimum number of Type-III
operations.

\printbibliography

\end{document}

